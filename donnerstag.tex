\renewcommand{\conferenceDay}{\donnerstag}
\newsmalltimeslot{09:00}

\abstractAPH{Otto Daussau}%
{GBD Web Suite}{}%
{In diesem Vortrag wird die neue GBD Web Suite vorgestellt, mit der Möglichkeit, Daten aus externen (Fach-)Anwendungen sowie mit QGIS aufbereitete Projekte zu integrieren und über die Komponenten GBD Web Server und GBD WebGIS Client darzustellen.}

% time: 2018-03-22 09:00:00
\abstractZwei{Jörg Thomsen}%
{Wie kommt der Schwimmbagger ins WebGIS}%
{}%
{%
Am Beispiel ein Unternehmens aus der Rohstoffbranche wird eine vollkommen automatisierten Datenverarbeitungskette aufgezeigt. Ein Schwimmbagger nimmt während er über den Baggersee schwimmt und baggert kontinuierlich seine Geoposition auf und misst gleichzeitig zu jeder Geokoordinate die Wassertiefe. Diese Daten sollen in einem WebGIS dargestellt und wöchentlich aktualisiert werden. Wenn es aber nicht nur einen Bagger auf einem See gibt, sondern viele Bagger auf vielen Seen ist eine automatische Verarbeitung der Daten bis hin zum WMS gefordert.%
}


% time: 2018-03-22 09:00:00
\abstractVier{Andreas Krumtung}%
{Potenziale und Herausforderungen eines Open Innovation-Ansatzes für offene Geo- und Vermessungsdaten der öffentlichen Verwaltung  }%
{}%
{%
2016 und der Veröffentlichung des Ersten Nationalen Aktionsplans im Sommer 2017 hat das Thema Open Government innerhalb der Verwaltungen Deutschlands große Bedeutung erlangt. Als Grundlage eines offenen Regierungs- und Verwaltungshandelns gelten vor allem offene Daten. Der Bund sowie einige Bundesländer haben dementsprechende Open-Data-, Transparenz- oder Informationsfreiheitsgesetze verabschiedet, die eine Öffnung von staatlichen Datenbeständen für die Allgemeinheit zum Ziel haben.%
}

\newtimeslot{09:35}
% time: 2018-03-22 09:35:00
\abstractAPH{Hans-Jörg Stark}%
{Lightning Talks}%
{}%
{%
siehe Aushang%
}

% time: 2018-03-22 09:35:00
\abstractZwei{Thomas Schüttenberg}%
{Jetzt in Ihrem QGIS: ISYBAU XML-Abwasserdaten}%
{in der Hauptrolle der OGR GMLAS Treiber}%
{%
Ein Arbeitsbericht über die Verwendung des OGR GMLAS Treibers für die Nutzung der „ISYBAU-Austauschformate Abwasser (XML)“ in QGIS 3. 
Ziel ist einerseits die Anzeige von Kanälen, Schächten und Abwasserbauwerken (möglichst) auf Knopfdruck, andererseits eine freie ISYBAU-Schnittstelle zu anderen abwasserbezogenen QGIS-Projekten und Werkzeugen, wie z.B. den QKan Plugins oder der schweizerischen Abwasserfachschale QGEP, die auf diese Weise auch für Anwender aus Deutschland interessant werden könnte.%
}

% time: 2018-03-22 09:35:00
\abstractVier{Christian Strobl}%
{CODE-DE - der nationale Zugang zu Copernicus-Daten für Deutschland}%
{}%
{%
Die Copernicus Data and Exploitation Platform – Deutschland (CODE-DE) ist der Nationale Copernicus Zugang für die Satellitendaten der Sentinel-Satellitenreihe und die Informationsprodukte der Copernicus Dienste. CODE-DE wird speziell Nutzern in Deutschland – von Behörden über Forschungseinrichtungen und Unternehmen bis hin zu Privatpersonen – einen einfachen und schnellen Zugang zu den Daten und Informationen aller operationellen Sentinel-Satelliten sowie der Copernicus Dienste ermöglich%
}


\newtimeslot{10:10}
% time: 2018-03-22 10:10:00
\abstractAPH{Arne Schubert}%
{YAGA}%
{Yet Another Geo Application}%
{%
Das YAGA Development Team stellt seinen finalen Release von leaflet-ng2 1.0.0, einer ganularen Integration von Leaflet in Angular 2 und folgende Versionen, vor. Es werden Vorteile und Modularität des Frameworks herausgestellt. Zudem werden weitere Module und die künftige Roadmap rund um das Framework vorgestellt.%
}

% time: 2018-03-22 10:10:00
\abstractZwei{Jörg Höttges}%
{QKan - QGIS Plugins zur Aufbereitung von Kanalnetzdaten für Simulationen}%
{Aktueller Stand und weitere Ziele}%
{%
QKan ist ein System aus QGIS-basierten Plugins, das zur Vor- und Nachbereitung von Daten zu kommunalen Entwässerungssystemen im Zusammenhang mit hydrodynamischen Simulationen dient. Die Daten werden in einer SpatiaLite-Datenbank gespeichert und können sowohl mit Hilfe der Plugins als auch mit den QGIS-Funktionen verarbeitet werden. Es werden der aktuelle Stand sowie die nächsten geplanten Entwicklungsschritte vorgestellt.%
}

% time: 2018-03-22 10:10:00
\abstractVier{Thomas Eiling}%
{Refaktorieren oder grüne Wiese?}%
{Die Reise von opencaching.de von einer Legacy Applikation zu Symfony Full Stack mit Responsive Webdesign.}%
{%
Seit dem März 2016 entwickelt das Team von OpenachingDeutschland an einer neuen Webseiten Version auf Basis von Symfony und Bootstrap. In diesem Vortrag möchten wir euch mit auf die Reise nehmen und über unsere erreichten Ziele und genommen Hürden und Hindernisse berichten die eine so große Legacy Applikation mit sich bringt.%
}


\newtimeslot{11:05}
% time: 2018-03-22 11:05:00
\abstractAPH{Katrin Hannemann}%
{Erstellung individueller Symbole in Inkscape für die Verwendung in QGIS}%
{}%
{%
Dieser Vortrag gibt einen Überblick über die Erstellung individueller SVG-Marker in Inkscape und deren Verwendung in QGIS. Dabei werden zunächst vorhandene Möglichkeiten in QGIS gezeigt. In Inkscape werden Benutzeroberfläche und wichtige Werkzeuge vorgestellt und erklärt, wie individuelle Symbole erzeugt werden können. Anschließend wird gezeigt, wie die individuell erstellten Symbole in QGIS verwendet und entsprechend angepasst werden können.
Abschließend wird gezeigt, welche Möglichkeiten es gibt, die neuen Symbole bereitzustellen und  mit der QGIS-Comm%
}

% time: 2018-03-22 11:05:00
\abstractZwei{Stefan Kuethe}%
{Dockerize stuff}%
{Postgis swarm and other geo boxes}%
{%
Dieser Talk zeigt wie man mit Docker und Docker Swarm in wenigen Schritten ein Postgis-Cluster mit einem Manager und mehreren Workern aufsetzt und mit einem GeoServer-Container verbindet. Darüber hinaus werden weitere Geo-Container (osrm, mapnik, mapshaper, ...) kurz vorgestellt.%
}

% time: 2018-03-22 11:05:00
\abstractVier{Hartmut Holzgraefe}%
{OSM Daten zu Papier bringen}%
{}%
{%
Es gibt viele  Online-Dienste die auf Basis von OSM-Daten schöne Karten generieren, aber nur sehr wenige davon eignen sich auch für Ausdrucke auf Papier. Mit MapOSMatic existiert eine Online-Lösung die diese Lücke schließen will.%
}

\newtimeslot{11:40}
% time: 2018-03-22 11:40:00
\abstractAPH{Peter Gipper}%
{Entwicklung von Plug-Ins für QGIS 3 - Eine Einführung}%
{}%
{%
Dieser Vortrag widmet sich der Entwicklung von QGIS 3 Plug-Ins und ist vor allem an Entwickler/innen oder Hobbyprogrammierer/innen gerichtet, die bereits QGIS 2 Plug-Ins programmiert haben. Die Entwicklung für eine Software, die noch in Entwicklung ist bzw. noch nicht etabliert ist, bringt einige Hürden mit sich und führt zu vielen Fragen. Dieser Vortrag geht speziell auf Änderungen ein, die beim Umstieg von QGIS 2 auf QGIS 3 relevant werden.%
}

% time: 2018-03-22 11:40:00
\abstractZwei{Volker Mische}%
{Noise}%
{Einfach Daten Durchsuchen}%
{%
Noise ist eine neue Bibliothek die dazu dient Daten im JSON Format zu durchsuchen. Eine einfache Handhabung, sowohl bei der Administration als auch bei der Datenabfrage ist zentrales Ziel. Der Vortrag gibt einen Überblick über die verwendeten Technologien Rust und RocksDB, und mündet in ein Live-Demo das u.a. die intuitive Abfragesprache vorstellt. Noise ist Open Source unter Apache 2.0/MIT Lizenz.%
}

% time: 2018-03-22 11:40:00
\abstractVier{Tobias Knerr}%
{3D Model Repository - Von der Parkbank bis zur Burg}%
{Freie 3D-Modelle für OpenStreetMap}%
{%
In OpenStreetMap werden zunehmend komplexe 3D-Modelle erstellt. Mit deren Detailgrad stoßen Mapper an die Grenzen dessen, was in OSM-Editoren sinnvoll zu bearbeiten ist. Wir haben daher eine offene Plattform zum Austausch frei lizenzierter Modelle geschaffen.

Dinge der realen Welt – von der Parkbank bis zur Burg – können in einem dafür ausgelegten 3D-Editor erstellt werden und über das 3D Model Repository von jedermann mit OpenStreetMap verknüpft werden.%
}

\newtimeslot{12:15}
% time: 2018-03-22 12:15:00
\abstractAPH{Marco Lechner}%
{Fortgeschrittene OpenLayers-Overlays im BfS Web-Client}%
{von der Visualisierung bis zum Druck}%
{%
Um den radiologischen Notfallschutz weiterzuentwickeln, setzt das Bundesamt für Strahlenschutz (BfS) auf eine Open-Source-Strategie. Im WebGIS des neuen IMIS3 werden OpenLayers, GeoExt und MapfishPrint eingesetzt und zur Weiterentwicklung der Projekte beigetragen. Der Vortrag präsentiert den fortgeschrittenen Einsatz von OpenLayers-Overlays im Web-Clienten von interaktiven Kartodiagrammen in denen Zeitreihen, Tabellen und Balkendiagramme dargestellt werden, bis zum Druck durch MapfishPrint3. Das BfS veröffentlicht Quellcode unter github.com/OpenBfS.%
}

% time: 2018-03-22 12:15:00
\abstractZwei{Pirmin Kalberer}%
{Styling und Publikation von Vektor-Tiles}%
{}%
{%
Vektor-Tiles haben das Potential die bewährten Rasterkarten in vielen Bereichen abzulösen oder mindestens massgeblich zu ergänzen. Für das Styling hat sich Mapbox GL JS als Industrie-Standard etabliert. Neben dem Viewer und den nativen SDK für Android, iOS, macOS, Node.js und Qt von Mapbox unterstützt auch OpenLayers den Import von Mapbox GL Styles.
Der Vortrag bietet eine Einführung in das Mapbox GL JS Styling Format und gibt Tipps zur Publikation von Vektor-Tiles.%
}

% time: 2018-03-22 12:15:00
\abstractVier{Tobias Knerr}%
{3D: Mehr als Gebäude}%
{OSM2World jenseits von Simple 3D Buildings}%
{%
Die Fähigkeit zur Darstellung von 3D-Gebäuden ist heute beinahe schon Standard. Für eine umfassende dreidimensionale Abbildung der Welt müssen aber auch viele andere Objekte berücksichtigt werden, und OpenStreetMap bietet dafür beste Voraussetzungen. Am Beispiel des freien 3D-Renderers OSM2World werden die Möglichkeiten der OSM-Daten für das 3D-Rendering jenseits von Gebäuden gezeigt.%
}


\newtimeslot{13:40}
% time: 2018-03-22 13:40:00
\abstractAPH{Felix Kunde}%
{PostGIS v2+}%
{Überblick an Funktionen der letzten Releases}%
{%
Mit jeder neuen Version unserer Lieblingsgeodatenbank PostGIS kommen neue spannende Funktionen hinzu. Auch das darunterliegende PostgreSQL entwickelt sich beständig weiter. Oft merkt man sich ein, zwei Highlights pro Release und übersieht bzw. vergisst den Rest. Dieser Vortrag lässt die neuen Features der einzelnen PostGIS-Releases seit der Version 2.0 im Jahr 2012 Revue passieren.%
}

% time: 2018-03-22 13:40:00
\abstractZwei{Numa Gremling}%
{Webmapping und Geoverarbeitung: Turf.js}%
{}%
{%
Turf.js ist eine Open Source JavaScript-Bibliothek die mit oft nur sehr wenigen Befehlen ermöglicht klassische Geoverarbeitungswerkzeuge im Browser auszuführen. Das Format GeoJSON ermöglicht das clientseitige Verarbeiten und Analysieren von Geodaten und spart Ihnen die Einrichtung einer komplexen serverseitigen Infrastruktur. Komfortabler geht es kaum: Turf einbinden, wenigen Zeilen Code schreiben und in sekundenschnelle komplexe ortsbezogene Fragen beantworten. Und das alle lokal in Ihrem Browser und sogar offline!%
}

% time: 2018-03-22 13:40:00
\abstractVier{Raffael }%
{Open Data im ÖPNV}%
{}%
{%
In der Präsentation wird zu Beginn ein Überblick über den Stand von Open Data im ÖPNV gegeben -- Schwerpunkt dabei sind Fahrplandaten in Deutschland. Im zweiten Teil der Präsentation werden Anwendungen vorgestellt die mit offenen ÖPNV-Daten arbeiten.%
}

\newtimeslot{14:15}
% time: 2018-03-22 14:15:00
\abstractAPH{Pirmin Kalberer}%
{GeoPackage als Arbeits- und Austauschformat}%
{}%
{%
In GeoPackage-Dateien können sowohl Vektor- als auch Rasterdaten mit zugehörigen Metainformation gespeichert werden. Damit können Geodaten einfach ausgetauscht und auch auf mobilen Geräten effizient genutzt werden.
Der Vortrag zeigt die Einsatzmöglichkeiten von GeoPackage mit dem Fokus auf QGIS und gibt einen aktuellen Überblick über GeoPackage-Extensions.%
}

% time: 2018-03-22 14:15:00
\abstractZwei{Christian Mayer}%
{Wegue - WebGIS-Anwendungen mit OpenLayers und Vue.js}%
{}%
{%
Wegue ist eine Open Source Software zum Erstellen von modernen leichtgewichtigen WebGIS-Client-Anwendungen. Die Basis dafür sind die beiden JavaScript-Frameworks OpenLayers und Vue.js.
Wegue verknüpft diese beiden Bibilotheken zu einer konfigurierbaren Vorlage für WebGIS-Anwendungen aller Art und stellt wiederverwendbare UI-Komponenten (z.B. Layer-Liste, FeatureInfo-Dialog, etc.) bereit.  Somit können Anwender und Entwickler schnell zu einem ansprechendem und modernen WebGIS-Klienten zur Veröffentlichung und Nutzung von Geodaten gelangen.%
}

% time: 2018-03-22 14:15:00
\abstractVier{Christoph Hormann}%
{Darstellungsorientierte Generalisierung von offenen Geodaten}%
{}%
{%
Dieser Vortrag stellt die jüngsten Entwicklungen im Bereich der darstellungsorientierten automatischen Generalisierung von offenen Geodaten vor.  Ziel hiervon ist es, die Qualität automatisiert regelbasiert produzierter Kartendarstellungen in digitalen Karten zu verbessern.   Anhand von Beispielen werden die Neuerungen und zusätzliche Anwendungsfelder vorgestellt und sowohl die Chancen als auch die Herausforderungen des darstellungsorientierten Ansatzes wie auch der Verwendung offener Geodaten erläutert.%
}

\newtimeslot{14:50}
% time: 2018-03-22 14:50:00
\abstractAPH{Felix Kunde}%
{Kompakte Datenbankschemata für dynamisch erweiterbare GML Application Schemas}%
{Die neue Version der 3DCityDB zeigt, wie es gehen kann}%
{%
Durch größere Verfügbarkeit von 3D-Geodaten wächst die Akzeptanz für CityGML und der Bedarf nach Domänen-spezifischen Erweiterungen des Standards (ADEs), z.B. Lärmkartierung oder Energiemanagement. Der Vortrag gibt einen Ausblick auf die neue Version der 3D City Database, die beliebige ADEs dynamisch einbinden kann, ohne dass das PostGIS-Datenbankschema zu komplex und schwerfällig wird.%
}

% time: 2018-03-22 14:50:00
\abstractZwei{ }%
{Adult.js - JavaScript ist erwachsen geworden!}%
{}%
{%
Die Zeiten, in denen JavaScript als reine Skriptsprache zur dynamischen Anpassung von HTML-Elementen in Browsern genutzt wurde, sind lange vorüber. Vielmehr werden mittlerweile komplexe Anwendungen mit JS programmiert, sowohl im Client als auch auf dem Server.

Der Vortrag gibt eine Übersicht über die heutigen Möglichkeiten der Geodatenverarbeitung im Client und Server mittels JavaScript. Außerdem wird die aktuelle Professionalisierung in der JavaScript-Entwicklung beleuchtet und bewertet.%
}

% time: 2018-03-22 14:50:00
\abstractVier{Thomas Skowron}%
{Pipelinebasierte Erzeugung von Karten}%
{Geodaten verarbeiten ohne Datenbanksystem}%
{%
Im OSM-Umfeld werden Daten meist erst in eine Datenbank geladen, um diese hiernach wieder zu extrahieren. Im Zuge dessen entstehen bei großen Datensätzen hierbei häufig Flaschenhälse, die eine effiziente Verarbeitung verhindern. Dieser Talk schlägt Methoden vor, um Daten sequentiell in einer Pipelinestruktur zu verarbeiten, um ressourcenschonend und schneller als bestehende Lösungen Daten zu verarbeiten, filtern und zu transformieren.%
}

\newtimeslot{15:45}
\abstractAPH{}{Lightning Talks}{}{}

\abstractZwei{Frederik Ramm}%
{Lügen mit Statistik, OpenStreetMap-Edition}%
{Missverständnisse und Fehlinterpretationen mit OSM-Metadaten}%
{%
In diesem Vortrag geht es nicht um die Geodaten in OpenStreetMap, sondern um die Daten hinter den Daten: Wer hat was wann eingetragen, wie viele Mapper arbeiten eigentlich an den Daten, und welche Daten sammeln die am liebsten? Immer wieder kommen Außenseiter hier zu drastischen Fehleinschätzungen. Dieser Vortrag zeigt ein paar richtige und falsche Statistiken und erklärt, wie man es richtig macht.%
}

% time: 2018-03-22 15:45:00
\abstractVier{Arndt Brenschede}%
{Energieeffizientes PKW Routing mit OpenStreetMap}%
{}%
{%
Energieeffizientes PKW Routing, manchmal auch Eco-Routing genannt, ist von der Idee nicht neu, aber kaum verbreitet und begrifflich undefiniert. Dieser Beitrag schafft hier Klarheit, zeigt das Potential für die Elektromobilität, diskutiert die besonderen Anforderungen, die energieeffizientes Routing an die Qualität von Straßenkarten stellt und untersucht die Eignung von OpenStreetMap für diesen Anwendungsbereich.%
}

\newtimeslot{16:20}
\abstractAPH{Johannes Kröger}%
{Karten aus QGIS ins Buch, Web oder auf die Leinwand}%
{Eine Übersicht der vielseitigen Exportmöglichkeiten von QGIS}%
{%
Neben den mitgelieferten Funktionen bietet QGIS dank seines umfangreichen Pluginkatalogs eine Vielzahl von Möglichkeiten Kartenprojekte in unterschiedlicher Art und Weise und für unterschiedlichste Zwecke zu exportieren. Etwa per automatisierter "Stapelverarbeitung", als interaktive Webkarten, Videos oder auch 3D-Viewer. Die Atlas-Erzeugung und die Plugins HTML Image Map Creator, qgis2web, QTiles, Time Manager und qgis2threejs stellen diese Optionen zur Verfügung.%
}

\abstractZwei{Robin Luckey}%
{Master Portal}%
{Das Open-Source Web-GIS der Stadt Hamburg}%
{%
Das Masterportal ist eine OGC-konforme, Open-Source (Mit-Lizenz) Web-GIS Lösung zur Generierung von digitalen Kartenanwendungen. Es basiert auf BackboneJS und Openlayers und wird aktiv von der Stadt Hamburg weiterentwickelt.
Es ermöglicht ohne Programmierkenntnisse und unter geringem Aufwand thematische Kartenanwendungen zu erstellen, außerdem ist es leicht Erweiterbar und kann es als Framework zur Erstellung von komplexen Kartenanwendungen genutzt werden.%
}

% time: 2018-03-22 16:20:00
\abstractVier{Michael Reichert}%
{Eisenbahnrouting mit GraphHopper}%
{}%
{%
In diesem Vortrag werden Anpassungen an GraphHopper vorgestellt, mit denen ein Routing auf Eisenbahngleisen möglich ist. Der Vortrag geht darauf ein, welche Anpassungen vorgenommen werden müssen und ist daher in Teilen auch als Anleitung zum Schreiben von FlagEncodern zu verstehen.

Ganz einfach ist das Routing auf Eisenbahngleisen jedoch nicht. Zwar wird jedes Gleis als ein Way in OSM erfasst, welches mit den anderen Gleisen verbunden ist. Manche Eigenschaften von Schienenfahrzeugen lassen sich nicht so einfach abbilden. Der Vortrag wird in seinem Ausblick daher kurz darlegen, was für ein besseres Routing noch fehlt.%
}


\newtimeslot{16:55}
% time: 2018-03-22 16:55:00
\abstractAPH{Marco Hugentobler}%
{Datenqualität sicherstellen mit QGIS}%
{}%
{%
QGIS bietet eine Reihe von Funktionen, um die Geometrien eines Datensatzes zu überprüfen und zu korrigieren. Der Vortrag gibt einen Ueberblick über die beiden Plugins 'Geometrychecker' und 'Topologychecker' und zeigt Gemeinsamkeiten und Unterschiede auf.%
}

% time: 2018-03-22 16:55:00
\abstractZwei{Martin Dresen}%
{BKG WebMap – ein OpenLayers 4 Framework zur einfachen Erstellung interaktiver Webkarten}%
{}%
{%
Die BKG WebMap des Bundesamts für Kartographie und Geodäsie ist eine JavaScript Bibliothek, die verschiedene Funktionen zur einfachen Erstellung interaktiver Karten bereithält. Sie wurde jetzt auf der Basis von OpenLayers 4 neu entwickelt und wird auf der FOSSGIS-Konferenz erstmalig präsentiert.%
}

% time: 2018-03-22 16:55:00
\abstractVier{Hartmut Holzgraefe}%
{OSM Daten mit Mapnik und Python rendern}%
{Eine kurze Einführung }%
{%
Mapnik ist eine Open Source Bibliothek zur Erstellung von Karten, wie zB. auf openstreetmap.org zu sehen. Mapnik bietet eine eigene XML-basierte Stylesheet-Sprache und verarbeitet Daten aus verschiedenen Geodaten-Quellen..%
}

\newtimeslot{17:30}
% time: 2018-03-22 17:30:00
\abstractAPH{Otto Dassau}%
{Geometrie- und Topologiefehler finden und korrigieren}%
{Möglichkeiten mit QGIS und GRASS GIS}%
{%
Ob man nun Daten selber generiert oder Daten von Anbietern verwendet. Man kommt nicht umhin, diese auf geometrische und topologische Fehler zu prüfen und diese zu bereinigen. 

In diesem Vortrag werden wir etwas hinter die Kulissen schauen. Welche Unterstützung bieten die Methoden der GEOS Bibliothek im Vergleich zu QGIS eigenen Algorithmen? Wir werden QGIS Plugins zur Geometrieprüfung vorstellen und deren Ergebnisse vergleichen und Hintergründe beleuchten. Alternativen werden aufgezeigt und andere Lösungswege skizziert, wie z.B. über das GRASS Plugin.%
}

% time: 2018-03-22 17:30:00
\abstractZwei{Armin Retterath}%
{INSPIRE Downloaddienste}%
{Praktische Erfahrungen der letzten 4 Jahre}%
{%
Im Vortrag werden die neuesten Entwicklungen bezüglich der Umsetzung und Nutzung von INSPIRE Downloaddiensten in den Ländern Hessen, Rheinland-Pfalz und Saarland anhand praktischer Beispiele vorgestellt. Dabei wird insbesondere der immense Mehrwert für die Praxis ersichtlich, den die Standardisierung durch die EU INSPIRE-Richtlinie gebracht hat.%
}

% time: 2018-03-22 17:30:00
\abstractVier{Petr Pridal}%
{OpenMapTiles}%
{Revolution in selbstgehosteten Karten}%
{%
Ihre eigenen weltweiten Straßenkarten auf einem lokalen Computer oder auf privaten und öffentlichen Clouds hosten? Ja, dank Vektorkacheln, Open-Source Software und Opendata. Erfahren Sie, wie Sie Karten mit eigenem Design in Ihren Websiten und mobilen Apps, oder in QGIS und ArcGIS anzeigen können. Generieren Sie eigene Vektor-Kacheln und hosten Sie diese selbst. Eigenen Geodaten können integriert werden. OpenMapTiles Projekt wird bereits von Siemens, IBM, Bosch, Amazon, SBB und anderen angewendet.%
}

\newsmalltimeslot{18:00}
\abstractZwei{}%
{FOSSGIS-Mitgliederversammlung}%
{Jährlich stattfindende Versammlung des FOSSGIS e.V.}%
{%
Alle Mitglieder sind herzlich eingeladen, teilzunehmen und sich zu beteiligen. Einige Themen stehen auf der Agenda. Wir laden ein zum Kennenlernen, zur Diskussion, Abstimmung & Neuwahlen. Es wird für alle Pizza bestellt und Getränke stehen bereit.%
}
